\begin{Schunk}
\begin{Sinput}
> library("Ham94", lib.loc = "../../../library")
\end{Sinput}
\end{Schunk}
Pages 409-410 gives a simple example of estimating the degrees of freedom of a standard t distribution. 
To illustrate, first generate a sample of 500 observations from a t distribution with 10 degrees of freedom.
\begin{Schunk}
\begin{Sinput}
> Y <- rt(500, 10)
\end{Sinput}
\end{Schunk}
Then maximize the sum of logs of a t density evaluated on the sample points.
\begin{Schunk}
\begin{Sinput}
> objective <- function(nu, Y) {
+     -sum(log(dt(Y, df = nu)))
+ }
> classical.results <- optimize(interval = c(1, 30), f = objective, 
+     Y = Y)
> mu2 <- mean(Y^2)
> nu <- 2 * mu2/(mu2 - 1)
> print(classical.results)
\end{Sinput}
\begin{Soutput}
$minimum
[1] 6.988635

$objective
[1] 769.8934
\end{Soutput}
\begin{Sinput}
> print(nu)
\end{Sinput}
\begin{Soutput}
[1] 6.536056
\end{Soutput}
\end{Schunk}
\subsection{Generalized Method of Moments}
Using the sample sample, we can estimate the degrees of freedom using GMM.  To this end define a
function following the GMM recipe in the text.  
\begin{Schunk}
\begin{Sinput}
> compute.estimates <- function(Y, h, interval) {
+     g <- function(Y, THETA) {
+         apply(X = apply(X = Y, MARGIN = 1, FUN = h, THETA = THETA), 
+             MARGIN = 1, FUN = mean)
+     }
+     objective <- function(THETA, Y, W) {
+         g.value <- g(Y, THETA)
+         t(g.value) %*% W %*% g.value
+     }
+     r <- length(h(Y[1, ], interval[[1]]))
+     a <- length(interval[[1]])
+     T <- dim(Y)[[1]]
+     stage.1.results <- optimize(interval = interval, f = objective, 
+         Y = Y, W = diag(r))
+     temp <- apply(X = Y, MARGIN = 1, FUN = h, THETA = stage.1.results$objective)
+     S <- 1/T * temp %*% t(temp)
+     stage.2.results <- optimize(interval = interval, f = objective, 
+         Y = Y, W = solve(S))
+     J.test <- 1 - pchisq(T * stage.2.results$objective, r - a)
+     list(stage.1.results = stage.1.results, stage.2.results = stage.2.results, 
+         overidentifying = J.test)
+ }
\end{Sinput}
\end{Schunk}
Using this function is then a matter of specifying an appropriate function h to define an observation of
the set of moments being targeted.
\begin{Schunk}
\begin{Sinput}
> h <- function(Yt, THETA) {
+     nu <- THETA
+     c(Yt^2 - nu/(nu - 2), Yt^4 - 3 * nu^2/((nu - 2) * (nu - 4)))
+ }
> estimates <- compute.estimates(Y %o% 1, h, interval = c(5, 30))
> print(estimates)
\end{Sinput}
\begin{Soutput}
$stage.1.results
$stage.1.results$minimum
[1] 5.000051

$stage.1.results$objective
          [,1]
[1,] 0.3187395


$stage.2.results
$stage.2.results$minimum
[1] 10.08829

$stage.2.results$objective
            [,1]
[1,] 0.002112142


$overidentifying
          [,1]
[1,] 0.3041131
\end{Soutput}
\end{Schunk}
A second example estimates the shape parameter of a two-sided gamma distribution.
\begin{Schunk}
\begin{Sinput}
> Yg <- rgamma(500, 10) * sign(runif(500, -1, 1))
> hg <- function(Yt, THETA) {
+     k <- THETA
+     nu <- k
+     mu <- k
+     sigma <- k
+     skew <- 2/sqrt(k)
+     kurt <- 6/k
+     c(Yt^2 - sigma - mu^2, Yt^4 - (kurt * (sigma^2) + 3) - 4 * 
+         (skew * sigma^1.5) * mu - 6 * sigma * mu^2 - mu^4)
+ }
> gestimates <- compute.estimates(Yg %o% 1, hg, interval = c(5, 
+     30))
> print(gestimates)
\end{Sinput}
\begin{Soutput}
$stage.1.results
$stage.1.results$minimum
[1] 9.867225

$stage.1.results$objective
          [,1]
[1,] 0.9806473


$stage.2.results
$stage.2.results$minimum
[1] 9.845532

$stage.2.results$objective
            [,1]
[1,] 0.000168614


$overidentifying
          [,1]
[1,] 0.7715434
\end{Soutput}
\end{Schunk}
\subsection{R Facilities for Generalized Method of Moments}
TBD
